\textbf{E1.} (2,5 puntos)


\vspace{20px}
\textit{Solución:}
\\

\begin{enumerate}
[label=\alph*)]
    \item Para resolver este problema se aplica el principio de superposición lineal. El problema se reduce así
    a sumar las contribuciones de una esfera con densidad $\rho_0$ y centro en O, y de otra esfera
    con densidad $-\rho_0$ y centro en O$^\prime$.

    Como los puntos A y B se sitúan en la superficie de las 2 esferas, se podría calcular el campo \textbf{E} mediante
    la aplicación de la ley de Gauss, o suponiendo que toda la carga se sitúa en el centro de la esfera y
    utilizando la expresión del campo eléctrico para una carga discreta.

    Con ambos métodos, los resultados para la esfera 1 (centrada en O) y para la esfera 2 (centrada en O$^\prime$) son:

    \begin{equation*}
        \mathbf{E}_1 = \frac{\rho_0 R}{3 \varepsilon_0} \mathbf{u}_{\rho}; \hspace{20pt}
        \mathbf{E}_2 = - \frac{\rho_0 R}{3 \varepsilon_0} \mathbf{u}_{\rho}
    \end{equation*}

    Solo existe componente radial del campo eléctrico para cada una de las esferas. Observando la simetría del
    problema, podemos expresar los campos eléctricos en los puntos A y B en coordenadas cartesianas:

    \begin{equation*}
        \mathbf{E}_A =  \mathbf{E}_B = \frac{\rho_0 R}{3 \varepsilon_0} (\mathbf{u}_x + \mathbf{u}_y )
    \end{equation*}


    \vspace{20px}
    \item awdfaef
    
    \begin{center}
        \begin{tikzpicture}
%    \draw[thin,gray!40] (-2,-2) grid (2,2);
            \draw[<->] (-2.5,0)--(2.5,0) node[right]{$x$};
            \draw[<->] (0,-2.5)--(0,2.5) node[above]{$y$};

            \draw[dashed] (-2,2)--(2,2) node[above,xshift=-2.4cm] {$d$};
            \draw[dashed] (-2,2)--(-2,-2) node[left,yshift=2.4cm] {$d$};;
            \draw[dashed] (-2,-2)--(2,2);



            \draw (-2,2) circle (6pt) node[above left=6pt]{$\boldsymbol{1}$};
            \draw (-2,2) pic[rotate = 0] {cross=4pt};

            \draw (2,2) circle (6pt) node[above right=6pt]{$\boldsymbol{2}$};
            \filldraw (2,2) circle (3pt);

            \draw (-2,-2) circle (6pt) node[below left=6pt]{$\boldsymbol{3}$};
            \filldraw (-2,-2) circle (3pt);


            \draw[line width=2pt,red,-stealth](0,0)--(-1,-1) node[anchor=east]{$\boldsymbol{\vec{B_1}}$};
            \draw[line width=2pt,blue,-stealth](0,0)--(0.5,-0.5) node[anchor=west]{$\boldsymbol{\vec{B_2}}$};
            \draw[line width=2pt,blue,-stealth](0,0)--(-0.5,0.5) node[anchor=south]{$\boldsymbol{\vec{B_3}}$};
        \end{tikzpicture}
    \end{center}

\end{enumerate}