\textbf{Q1.} (0,5 puntos)


\vspace{20px}
\textit{Solución:}
\\

Una de las propiedades más importantes del campo electrostático, así como de todos los campos conservativos,
es que la integral de línea del campo entre dos puntos no depende del camino, o dicho de otro modo, la
integral de un campo electrostático a lo largo de un camino cerrado es nula.

\begin{equation*}
    \oint_{C} \textbf{E} \cdot d\textbf{l} = 0
\end{equation*}

Por otro lado, sabemos por el Teorema de Stokes que la integral del rotacional de un campo vectorial \textbf{F}
sobre una superficie abierta
$S$ es igual a la integral del campo \textbf{F} a lo largo del camino $C$ que limita $S$.

\begin{equation*}
    \int_{S} (\nabla \times \textbf{F}) \cdot d\textbf{s} = \oint_{C} \textbf{F} \cdot d\textbf{l}
\end{equation*}

Como esta relación se debe cumplir para cualquier camino $C$ que limite una superficie $S$, podemos concluir que, para que
\textbf{F} sea un posible campo electrostático, su
rotacional $\nabla \times \textbf{F}$ debe ser nulo.

Aplicando la ecuación del rotacional en coordenadas esféricas e igualando el resultado a 0 obtendremos los valores de
$A$, $B$ y $C$ que hacen que se cumpla esta condición.

\renewcommand{\arraystretch}{1.2}

\begin{align*}
    \nabla \times \textbf{F}
    =\; &
    \frac{1}{r^2 \sen \theta}
    \begin{vmatrix}
        \mathbf{u}_{r}        & r\mathbf{u_{\theta}}       & (r\sen\theta)\mathbf{u_{\varphi}} \\
        \partial / \partial r & \partial / \partial \theta & \partial / \partial \varphi       \\
        F_r                   & r F_\theta                 & (r \sen\theta)F_\varphi           \\
    \end{vmatrix}
    \\[10pt]
    =\; &
    \frac{1}{r^2 \sen \theta}
    \begin{vmatrix}
        \mathbf{u}_{r}                      & r\mathbf{u_{\theta}}                 & (r\sen\theta)\mathbf{u_{\varphi}} \\
        \partial / \partial r               & \partial / \partial \theta           & \partial / \partial \varphi       \\
        \frac{1}{r^4}(A + B \sen^2{\theta}) & \frac{1}{r^3}C \sen\theta \cos\theta & 0                                 \\
    \end{vmatrix}     \\[10pt]
    =\; &   \frac{1}{r^2 \sen \theta} \Biggl[
        - \frac{\partial}{\partial \varphi} \Bigl( \frac{1}{r^3} C \sen\theta \cos\theta \Bigr) \mathbf{u}_{r}
        + \frac{\partial}{\partial \varphi} \Bigl( \frac{1}{r^4} (A + B \sen^2{\theta})  \Bigr) r\mathbf{u}_{\theta} \; +  \\[10pt]
        & \biggl( \frac{\partial}{\partial r} \Bigl( \frac{1}{r^3} C \sen\theta \cos\theta   \Bigr)
        -  \frac{\partial}{\partial \theta} \Bigl( \frac{1}{r^4} (A + B \sen^2{\theta})
        \Bigr) \biggr) r \sen\theta \mathbf{u}_{\varphi}
        \Biggr] \\[10pt]
    =\; &  \frac{1}{r}
    \Biggl[
        \frac{\partial}{\partial r}
        \Bigl( \frac{1}{r^3} C \sen\theta \cos\theta
        \Bigr)
        -
        \frac{\partial}{\partial \theta}
        \Bigl( \frac{1}{r^4} (A + B \sen^2{\theta})
        \Bigr)
        \Biggr] \mathbf{u}_{\varphi}\\[10pt]
    =\; &  \frac{1}{r}
    \Biggl[
        \frac{(-3)}{r^4} C \sen\theta \cos\theta
        -
        \frac{1}{r^4} 2 B \sen\theta \cos\theta
        \Biggr] \mathbf{u}_{\varphi}\\[10pt]
    =\; &
    \frac{(- \sen\theta \cos\theta)}{r^5} (3C + 2B)  \mathbf{u}_{\varphi}  = 0 \\[10pt]
\end{align*}


Tras este cálculo, llegamos a la conclusión de que para que el rotacional del campo \textbf{F} sea idénticamente 0,
la constante $A$ puede tener cualquier valor,
y las constantes $B$ y $C$ deben cumplir la relación $C = -2 B / 3$.

Podríamos haber adivinado que la constante $A$ no está restringida a ningún valor porque su efecto es constante en todas
las direcciones del sistema de coordenadas. A lo largo de cualquier camino cerrado su contribución se anula.






