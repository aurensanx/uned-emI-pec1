\textbf{Q2.} (0,5 puntos)


\vspace{20px}
\textit{Solución:}
\\

La figura representa una distribución discreta de cargas puntuales, ya que hemos supuesto los excesos de carga se
acumulan en puntos centrados en cada átomo.

El momento dipolar para una distribución discreta de cargas es el sumatorio de la contribución de cada carga:

\begin{equation*}
    \mathbf{p} = \sum_{i = 1}^{N} q_i \mathbf{r}_i
\end{equation*}

Si suponemos que el origen de coordenadas cartesiano se sitúa en el centro del átomo de azufre, y tras calcular el momento
dipolar con la fórmula anterior, igualamos el módulo del vector resultante
al valor que se nos da el enunciado, podremos obtener un valor para el exceso de carga $\delta^-$.

\begin{align*}
    \mathbf{p} =\; &
    0\delta^+ + \sen(\frac{119\degree}{2}) \cdot 143,1 pm\; \delta^- \mathbf{u}_x
    - \cos(\frac{119\degree}{2}) \cdot 143,1 pm\; \delta^- \mathbf{u}_y\\
    & - \sen(\frac{119\degree}{2}) \cdot 143,1 pm\; \delta^- \mathbf{u}_x
    - \cos(\frac{119\degree}{2}) \cdot 143,1 pm\; \delta^- \mathbf{u}_y \\
    =\; & (-2) \cos(\frac{119\degree}{2}) \cdot 143,1 pm\; \delta^- \mathbf{u}_y \
\end{align*}

Por la simetría de la molécula, las componentes del momento dipolar en la dirección $\mathbf{u}_x$ se anulan. Igualando
el módulo de la expresión anterior al valor del momento dipolar dado en el enunciado y convirtiendo las unidades, obtenemos
el exceso de carga $\delta^-$ de un átomo de oxígeno.


\begin{align*}
    \delta^- = \frac{1,633 D}{(-2) \cos(\frac{119\degree}{2}) \cdot 1,431 \times 10^{-10}m\;}
    \cdot \frac{(299\;792\;458)^{-1} \times 10^{-21}C\;m}{1D} \cdot \frac{1e}{1,602 \times 10^{-19}C} = -0,234e
\end{align*}

Cada átomo de oxígeno tiene un exceso de carga $\delta^- = -0,234e$, y como la carga total del sistema es 0, el exceso de carga
$\delta^+$ del átomo de azufre es $+0,468e$.