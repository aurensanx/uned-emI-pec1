\textbf{Q3.} (0,5 puntos)


\vspace{20px}
\textit{Solución:}
\\

El campo eléctrico \textbf{E} en el interior del condensador cilíndrico solo tiene componente radial, y no es uniforme, siendo mayor
cuanto más cerca nos encontremos del radio interior. Despreciamos los efectos de borde porque la altura del condensador es idealmente infinita.

Podemos calcular el campo eléctrico en el espacio entre los dos conductores mediante el teorema de Gauss. A continuación, calculamos
la expresión del potencial eléctrico $V_0$ a partir del campo $\mathbf{E}$.

Sabiendo que $E_{m\acute ax} = 3 \times 10^6\;V / m$ para que no haya ruptura dieléctrica, podremos calcular el máximo potencial posible que se
puede aplicar al condensador.

Suponemos entonces que existe una carga $Q$ en el radio interior del condensador y una carga $-Q$ en el radio exterior. Si imaginamos una
superficie de Gauss cilíndrica alrededor del radio interior del condensador y aplicamos la ley de Gauss, obtenemos:

\begin{equation*}
    \oint_{S} \mathbf{E} \cdot d\mathbf{s} = 2 \pi \rho h E_{\rho} = \frac{Q}{\varepsilon_0}
    \hspace{10pt} \Rightarrow \hspace{10pt}  \mathbf{E} = \frac{Q}{2 \pi \varepsilon_0 h}\cdot \frac{1}{\rho} \mathbf{u}_{\rho}
\end{equation*}

Obtenemos ahora la expresión para el potencial eléctrico $V_0$:

\begin{equation*}
    V_0 = V_{a} - V_{b} = - \int_{b}^{a} \mathbf{E} \cdot d\mathbf{l} =
    - \int_{b}^{a} \biggl( \frac{Q}{2 \pi \varepsilon_0 h}\cdot \frac{1}{\rho} \mathbf{u}_{\rho} \biggr) \cdot \mathbf{u}_{\rho} d\rho
    =  \frac{Q}{2 \pi \varepsilon_0 h} \ln(\frac{b}{a})
\end{equation*}

Con estos dos resultados, obtenemos una relación entre $V_0$ y \textbf{E}:

\begin{equation*}
    \mathbf{E} = \frac{V_0}{ln(b/a)} \cdot \frac{1}{\rho} \mathbf{u}_{\rho}
\end{equation*}


La relación entre $V_0$, $a$ y $b$ que se nos pide en el enunciado es:

\begin{equation*}
    V_{0} < a\;\ln({\frac{b}{a}})\;E_{m \acute ax}
\end{equation*}

Esta inecuación tiene sentido físico si pensamos que el voltaje máximo permitido depende
tanto del radio interior como exterior, lo que no debería sorprendernos. A su vez, si aumentamos el radio
exterior $b$, el voltaje máximo aumenta. También nos damos cuenta de que si aumentamos el radio interior $a$ y mantenemos la misma
separación con el radio exterior $b$, el voltaje máximo aumenta de nuevo. En esencia, lo que hemos hecho en ambos casos
hipotéticos es aumentar
el tamaño del condensador, que así soporta un mayor voltaje.

Sustituyendo para los valores numéricos dados:

\begin{equation*}
    V_{0\;m\acute ax} = 5 \times 10^{-3}m \cdot \ln({\frac{7 \times 10^{-3}m}{5 \times 10^{-3}m}}) \cdot  3 \times 10^6\;V / m
    = 5,047kV
\end{equation*}

Si se aplica al condensador un potencial mayor que $5,047kV$ se producirá la ruptura.



